
% Definições para glossario

% ATENCAO o SHARELATEX GERA O GLOSSARIO/LISTAS DE SIGLAS SOMENTE UMA VEZ
% CASO SEJA FEITA ALGUMA ALTERAÇÃO NA LISTA DE SIGLAS OU GLOSSARIO É NECESSARIO UTILIZAR A OPÇÃO :
% "Clear Cached Files" DISPONIVEL NA VISUALIZAÇÃO DOS LOGS 
% ---
% https://www.sharelatex.com/learn/Glossaries

\newglossaryentry{COVID-19} {
    name=COVID-19,
    description={Infecção respiratória aguda causada pelo coronavírus SARS-CoV-2.}
}

\newglossaryentry{AMPARA} {
    name=AMPARA Animal,
    description={\ac{OSCIP} sem fins lucrativos que ajuda abrigos e protetores independentes com ração, medicamentos e atendimento veterinário.}
}

\newglossaryentry{Miau-dorei} {
    name=Mi-au-dorei,
    description={Termo definido para quando um usuário demonstra interesse em adotar um animal, semelhante ao like (termo em inglês para gostei).}
}

\newglossaryentry{Des-au-gostei} {
    name=Des-au-gostei,
    description={Termo definido para quando um usuário não gostou do perfil de um animal e/ou não tem interesse em adotar o animal do perfil que está interagindo, semelhante ao dislike (termo em inglês para desgostei).}
}

\newglossaryentry{Match} {
    name=Match,
    description={Termo definido para simbolizar o miau-dorei mútuo entre tutor do animal e adotante.}
}

\newglossaryentry{CSharp} {
    name=C\#,
    description={Linguagem de programação orientada a objetos criada pela Microsoft.}
}

\newglossaryentry{Java} {
    name=Java,
    description={Linguagem de programação orientada a objetos parte do núcleo da Plataforma Java.}
}

\newglossaryentry{Discord} {
    name=Discord,
    description={Aplicativo de comunicação por voz ou texto.}
}

\newglossaryentry{WhatsApp} {
    name=WhatsApp,
    description={Serviço de mensagens e chamadas.}
}

\newglossaryentry{Azure} {
    name=Azure,
    description={Plataforma da Microsoft destinada à execução de aplicativos e serviços, baseada nos conceitos da computação em nuvem.}
}

\newglossaryentry{YouTube} {
    name=YouTube,
    description={Plataforma de compartilhamento de vídeos.}
}

\newglossaryentry{Bootstrap} {
    name=Bootstrap,
    description={Framework de front-end para desenvolvimento web.}
}

\newglossaryentry{PostgreSQL} {
    name=PostgreSQL,
    description={Sistema de gerenciamento de banco de dados objeto-relacional.}
}

\newglossaryentry{PostGIS} {
    name=PostGIS,
    description={Extensão espacial de código livre do \gls{PostgreSQL}.}
}

\newglossaryentry{VSCode} {
    name=Visual Studio Code,
    description={Editor de código destinado ao desenvolvimento de aplicações web.}
}

\newglossaryentry{Eclipse} {
    name=Eclipse,
    description={Ambiente de desevolvimento multilinguagem, mais popularmente usada para Java.}
}

\newglossaryentry{Latex} {
    name=\LaTeX,
    description={Sistema de preparação de documentos baseado no uso de texto simples.}
}

\newglossaryentry{Subversion} {
    name=Subversion,
    description={Sistema de controle de versão open-source.}
}

\newglossaryentry{Gource} {
    name=Gource,
    description={Ferramenta de visualização de controle de versão.}
}

\newglossaryentry{CodeIgniter} {
    name=CodeIgniter,
    description={Framework de código aberto para desenvolvimento de aplicações web com \ac{PHP}.}
}

\newglossaryentry{Heroku} {
    name=Heroku,
    description={Plataforma de desenvolvimento e execução de aplicações na nuvem.}
}

\newglossaryentry{000Webhost} {
    name=000Webhost,
    description={Plataforma gratuita de criação e hospedagem de sites.}
}

\newglossaryentry{Loom} {
    name=Loom,
    description={Ferramenta para fazer vídeos, gravando a tela do computador e a voz, conforme a necessidade}
}

\newglossaryentry{Instagram} {
    name=Instagram,
    description={Rede social de fotos para usuários de Android e iPhone.}
}

\newglossaryentry{Hostname} {
    name=Hostname,
    description={É um rótulo que identifica um dispositivo de hardware ou hospedeiro.}
}

\newglossaryentry{GitHub} {
    name=GitHub,
    description={É uma plataforma de hospedagem de código-fonte e arquivos com controle de versão usando o Git.}
}

\newglossaryentry{Google Meet} {
    name=Google Meet,
    description={Serviço de comunicação por vídeo desenvolvido pelo Google.}
}

\newglossaryentry{TortoiseSVN} {
    name=TortoiseSVN,
    description={TortoiseSVN é um cliente do Subversion para Microsoft Windows.}
}

\newglossaryentry{E-mail} {
    name=E-mail,
    description={Correio eletrônico; recurso que torna possível o envio e recebimento de mensagens pela Internet.}
}

% Normalmente somente as palavras referenciadas são impressas no glossario, portanto é necessário referenciar utilizando \gls{identificação}                
