\chapter{Resultados obtidos}
Neste capítulo demonstramos os resultados obtidos e evolução de nosso projeto a partir de dados e estatísticas coletados ao longo dos bimestres que se passaram. Nossos objetivos foram consideravelmente alcançados, a começar pela \autoref{tab-metricas}, com o título auto-explicativo, nela está contida de forma geral a evolução de algumas informações a respeito do projeto. A partir de tais dados é perceptível que a equipe de fato esteve muito focada todos esses meses a realmente manter o projeto avançando constantemente. Dito isso, foi necessário que assegurássemos o controle das informações, agora incluso no atual capítulo.

% Falar dos quesitos de seguranca e internacionalizacao e da evolucao (incluir tabela de metricas)

 



\begin{table}[htb]
\centering
\caption{Evolução das métricas}
\label{tab-metricas}
\begin{tabular}{p{5.7cm}cccccccc}
    \hline
\thead{Item} & \thead{Maio} & \thead{Jun}  & \thead{Jul} & \thead{Ago} & \thead{Set} \\
    \hline
Arquivos                                          & 1    & 20     & 32    & 39       &47   \\
Atributos                                         & 0    & 9      & 64    & 77       & 82    \\
Classes                                             & 0    & 1      & 11     & 12      &20    \\
Commits                                      & 2    & 14     & 28  &42       & 43              \\
Dados sobre análises estáticas                    & 0    & 0      & 0   &0       & 1            \\
Entidades de banco de dados & 0    & 8      & 11  &13       &17                                    \\
Imagens                                           & 0    & 0      & 1     & 2       & 2            \\
Interfaces                                        & 0    & 7      & 11     & 10      &22           \\
Linhas                                            & 0    & 610    & 1198    & 2633      &4457         \\
Métodos                                           & 0    & 9      & 34     & 43       &117            \\
Postagens do Blog                                 & 2    & 7      & 11   & 16   & 20               \\
Requisitos                                        & 0    & 18     & 16  & 16       & 16         \\
Reuniões                                          & 4    & 14     & 23   & 25       & 26            \\
Sons                                            & 0    & 0      & 0   & 0       & 0            \\
Tamanho do Projeto \textit{(MB)}         & 0    & 0 & 2     & 3.6       & 4.3           \\
Testes                                            & 0    & 1      & 1   & 1       & 3          \\
Testes Unitários/automatizados                  & 0    & 0      & 0   & 0       & 1           \\
Vídeos gerados                                    & 0    & 1      & 3   & 4       & 4         \\
    \hline
\end{tabular}
\fonte{Elaborado pelos autores}
\end{table}

Todo o processo de desenvolvimento da aplicação foi registrado em \textit{posts} semanais no blog da equipe (\autoref{publicacoes-blog}); em vídeos no canal do \gls{YouTube} (\autoref{qr-url-2}); no repositório \autoref{qr-rep}; e o \textir{site} pode ser acessado em (\autoref{qr-url-1}).

\begin{figure}[htb]
\caption{\label{qr-url-1}URL da aplicação}
\begin{pspicture}(25mm,25mm)
\psbarcode{https://petinderapp.azurewebsites.net/}{eclevel=H width=1.0 height=1.0}{qrcode}
\end{pspicture}
\legend{\url{https://petinderapp.azurewebsites.net/}}
\fonte{Elaborado pelos autores}
\end{figure}

\begin{figure}[htb]
\caption{\label{qr-rep}Repositório SVN}
\begin{flushright}
\begin{pspicture}(25mm,25mm)
\psbarcode{https://svn.spo.ifsp.edu.br/svn/a6pgp/A2021-PDS431/TI.TI.TI/}{eclevel=H width=1.0 height=1.0}{qrcode}
\end{pspicture}
\legend{\url{https://svn.spo.ifsp.edu.br/svn/a6pgp/A2021-PDS431/TI.TI.TI/}}
\fonte{Elaborado pelos autores}
\end{flushright}
\end{figure}

\begin{figure}[htb]
\caption{\label{qr-url-2}Canal da equipe no Youtube}
\begin{pspicture}(25mm,25mm)
\psbarcode{https://www.youtube.com/channel/UCbgWTXMmWO9rkkGh8TA2l8Q}{eclevel=H width=1.0 height=1.0}{qrcode}
\end{pspicture}
\legend{\url{https://www.youtube.com/channel/UCbgWTXMmWO9rkkGh8TA2l8Q}}
\fonte{Elaborado pelos autores}
\end{figure}

\begin{figure}[htb]
\caption{\label{qr-blog}Blog da equipe}
\begin{flushright}
\begin{pspicture}(25mm,25mm)
\psbarcode{https://equipetititi.blogspot.com/}{eclevel=H width=1.0 height=1.0}{qrcode}
\end{pspicture}
\legend{\url{https://equipetititi.blogspot.com/}}
\fonte{Elaborado pelos autores}
\end{flushright}
\end{figure}

% ---
% Conclusão (outro exemplo de capítulo sem numeração e presente no sumário)
% Dependendo do trabalho desenvolvido ele pode ter uma Conclusão ou Considerações finais
% Para trabalhos de disciplina utilizar Considerações Finais
% ---
\chapter{Considerações Finais}
O ano letivo de 2021 foi altamente imprevisível e exaustivo para todas as integrantes da equipe TI TI TI. Finalizar o último ano de um ensino técnico integrado ao ensino médio no método de Ensino Remoto Emergencial e realizar um projeto como o solicitado pela matéria de \ac{PDS} exigiu muita maturidade e crescimento pessoal.

A conclusão do PETINDER demandou coisas como a habilidade de trabalhar em grupo e respeitar os colegas de equipe (no sentido de não faltar com responsabilidade e sobrecarregar outro integrante), a capacidade de organização, envolvendo equilibrar as demais matérias do \ac{IFSP}, vida pessoal, preparação para o vestibular e o desenvolvimento do projeto e da documentação. Foi necessário o aprimoramento de capacidades técnicas relacionadas a programação que iam além do conteúdo ministrado em aula, nos forçando a buscar sempre mais conhecimento em fontes externas.

Com o tempo a equipe se aprimorou, reconhecemos que o aprendizado resultante dessas dificuldades será uma grande ajuda no nosso futuro profissional e no nosso desenvolvimento como cidadãs contribuintes em uma sociedade. Aprendemos que nem sempre o que foi planejado no início será executado com perfeição, mas que há espaço para melhorias, aprendemos também que quando direcionamos um integrante para a área com a qual ele tem mais afinidade, o resultado tenderá a ser melhor. Em conclusão, saímos dessa experiência gratas e mais preparadas para o futuro.

