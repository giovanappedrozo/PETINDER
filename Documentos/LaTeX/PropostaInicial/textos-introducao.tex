% ----------------------------------------------------------
% Introdução
% ----------------------------------------------------------
\chapter[Introdução]{Introdução}
Este documento tem o objetivo de apresentar e justificar a proposta da criação da aplicação web PETINDER. Será apresentada a ideia do projeto, suas funcionalidades e as tecnologias que serão utilizadas em seu desenvolvimento no decorrer do ano. Além disso, serão apresentados protótipos de baixa fidelidade das telas e um fluxograma de dados da aplicação.

\section{Objetivos}
A aplicação web PETINDER é um sistema de adoção de cães e gatos que visa facilitar o processo de adoção desses animais. O sistema funciona de forma similar ao aplicativo de namoro, já existente, \textit{Tinder}, e busca conectar pessoas que querem adotar um ou mais animais com animais disponíveis para adoção, poupando o tempo de adotantes e doadores, além de otimizar o processo da adoção de forma que se torne fácil e com isso, talvez, o número de animais abandonados possa diminuir.

\section{Justificativa}
O projeto PETINDER surgiu após a apuração de dados sobre os índices de abandono de cães e gatos no Brasil, que, segundo a Organização Mundial da Saúde, existem cerca de 30 milhões de animais abandonados no país, onde, aproximadamente, 10 milhões são gatos, e 20 cachorros \cite{animais_abandonados}. Apenas na cidade de São Paulo, estima-se a existência de 2 milhões de animais abandonados pelas ruas \cite{sao_paulo}.\\
Com todos esses dados em mãos, o PETINDER tem como foco, facilitar o processo de adoção de cães e gatos, prevenindo o abandono.




