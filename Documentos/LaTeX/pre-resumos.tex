% ---
% RESUMOS
% ---

% resumo em português
\setlength{\absparsep}{18pt} % ajusta o espaçamento dos parágrafos do resumo
\begin{resumo}
\label{resumo}
Este documento, contendo textos, arquivos e imagens, foi formulado de modo a detalhar o desenvolvimento da aplicação \textit{web} PETINDER, solicitado pela disciplina de \ac{PDS}. O conceito deste projeto está relacionado ao crescente problema social de abandono animal, mesmo considerando que no início da pandemia de \gls{COVID-19} os números de adoção tenham aumentado no Brasil, o dilema
persistiu e a taxa de animais em situação de desamparo voltou a subir. A plataforma PETINDER visa facilitar o processo de adoção conectando de forma dinâmica doador e adotante. O usuário que optar pela nossa aplicação irá desfrutar de filtros de preferência quanto a características do animal que deseja adotar, uma listagem de animais cadastrados no sistema, poderá interagir com os perfis disponíveis (por meio do \gls{Miau-dorei} e do \gls{Des-au-gostei}, que funcionam respectivamente como nossas opções de gostei e não gostei) entre outras funções, como \textit{chat}, localização e \gls{Match}, que serão detalhadas ao longo da documentação.
%Segundo a ABNT(2003, 3.1-3.2), o resumo deve ressaltar o contexto, o objetivo, o método,os resultados e as conclusões do documento (portanto deve ser escrito por ultimo). A ordem e a extensão destes itens dependem do tipo de resumo (informativo ou indicativo) e do  tratamento que cada item recebe no documento original. O resumo \textbf{deve ter um paragrafo único} e deve \textbf{ter entre 150 e 500 palavras para trabalhos acadêmicos ou entre 100 e 250 para artigos de periódicos}. O resumo deve ser  precedido da referência do documento, com exceção do resumo inserido no  próprio documento. (\ldots) As palavras-chave devem figurar logo abaixo do resumo, antecedidas da expressão \textbf{Palavras-chave}:, separadas entre si por ponto e finalizadas também por ponto.

 \textbf{Palavras-chaves}: Adoção. Animais. Aplicação web. Doação.
\end{resumo}

% resumo em inglês
\begin{resumo}[Abstract]
 \begin{otherlanguage*}{english}
   This document, containing texts, files and images, was formulated in order to detail the development of the web application PETINDER, requested by the discipline of \ac{PDS}. The concept of this project is related to the growing social problem of animal abandonment, even considering that at the beginning of the \gls{COVID-19} pandemic the adoption numbers increased in Brazil, the problem persisted, and the abandonment rate rose again. The PETINDER platform aims to facilitate the adoption process by dynamically connecting donor and adopter. The user who chooses our application will be able to use preferential filters regarding the characteristics of the animal he wishes to adopt, a list of animals registered in the system, he will be able to interact with the available profiles (using \gls{Miau-dorei} and \gls{Des-au-gostei}, which work respectively as our like and dislike options) among other functions such as chat, localization and \gls{Match}, which will be detailed throughout the documentation.


   \vspace{\onelineskip}

   \noindent 
   \textbf{Keywords}: Adoption. Animals. Web aplication. Donation.
 \end{otherlanguage*}
\end{resumo}