% ---
% Conclusão (outro exemplo de capítulo sem numeração e presente no sumário)
% Dependendo do trabalho desenvolvido ele pode ter uma Conclusão ou Considerações finais
% Para trabalhos de disciplina utilizar Considerações Finais
% ---
\chapter{Considerações Finais}
O ano letivo de 2021 foi altamente imprevisível e exaustivo para todas as integrantes da equipe TI TI TI. Finalizar o último ano de um ensino técnico integrado ao ensino médio no método de \ac{EAD} e realizar um projeto como o solicitado pela matéria de \ac{PDS} exigiu muita maturidade e crescimento pessoal.

A conclusão do PETINDER demandou coisas como a habilidade de trabalhar em grupo e respeitar os colegas de equipe (no sentido de não faltar com responsabilidade e sobrecarregar outro integrante), a capacidade de organização, envolvendo equilibrar as demais matérias do \ac{IFSP}, vida pessoal, preparação para o vestibular e o desenvolvimento do projeto e da documentação. Foi necessário o aprimoramento de capacidades técnicas relacionadas a programação que iam além do conteúdo ministrado em aula, nos forçando a buscar sempre mais conhecimento em fontes externas.

Com o tempo a equipe se aprimorou, reconhecemos que o aprendizado resultante dessas dificuldades será uma grande ajuda no nosso futuro profissional e no nosso desenvolvimento como cidadãs contribuintes em uma sociedade. Aprendemos que nem sempre o que foi planejado no início será executado com perfeição, mas que há espaço para melhorias, aprendemos também que quando direcionamos um integrante para a área com a qual ele tem mais afinidade, o resultado tenderá a ser melhor. Em conclusão, saímos dessa experiência gratas e mais preparadas para o futuro.

